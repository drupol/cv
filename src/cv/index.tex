% Preamble
%%%%%%%%%%%%%%%%%%%%%%%%%%%%%%%%%%%%%%%%%
% Developer CV
% LaTeX Template
% Version 1.0 (28/1/19)
%
% This template originates from:
% http://www.LaTeXTemplates.com
%
% Authors:
% Jan Vorisek (jan@vorisek.me)
% Based on a template by Jan Küster (info@jankuester.com)
% Modified for LaTeX Templates by Vel (vel@LaTeXTemplates.com)
%
% License:
% The MIT License (see included LICENSE file)
%
%%%%%%%%%%%%%%%%%%%%%%%%%%%%%%%%%%%%%%%%%

% See also
% https://github.com/jankapunkt/latexcv

%----------------------------------------------------------------------------------------
%	PACKAGES AND OTHER DOCUMENT CONFIGURATIONS
%----------------------------------------------------------------------------------------

\documentclass[9pt]{developercv} % Default font size, values from 8-12pt are recommended

\usepackage{xstring}
\usepackage{catchfile}

\newcommand{\gitfolder}{.git}
\CatchFileDef{\headfull}{\gitfolder/HEAD}{}
\StrGobbleRight{\headfull}{1}[\head]
\StrBehind[2]{\head}{/}[\branch]
\CatchFileDef{\commit}{\gitfolder/refs/heads/\branch}{}
\StrGobbleRight{\commit}{1}[\longcommithash]

\newcommand{\gitrevisionlong}{%
	\longcommithash%
}

\newcommand{\gitrevisionsmall}{%
	\StrLeft{\longcommithash}{7}%
}


%----------------------------------------------------------------------------------------
% CUSTOM STRUT FOR EMPTY BOXES
%----------------------------------------- -----------------------------------------------
\newcommand{\mystrut}{\rule[-.3\baselineskip]{0pt}{\baselineskip}}


\begin{document}

%----------------------------------------------------------------------------------------
%	TITLE AND CONTACT INFORMATION
%----------------------------------------------------------------------------------------

\begin{minipage}[t]{0.5\textwidth} % 45% of the page width for name
	\vspace{-\baselineskip} % Required for vertically aligning minipages

	% If your name is very short, use just one of the lines below
	% If your name is very long, reduce the font size or make the minipage wider and reduce the others proportionately
    {\HUGE\textcolor{black}{\textbf{Pol}}} % First name

	{\HUGE\textcolor{black}{\textbf{Dellaiera}}} % Last name
    \vspace{6pt}

	{\huge Research, analysis, development} % Career or current job title
\end{minipage}
\begin{minipage}[t]{0.30\textwidth} % 27.5% of the page width for the first row of icons
	\vspace{-\baselineskip} % Required for vertically aligning minipages

	% The first parameter is the FontAwesome icon name, the second is the box size and the third is the text
	% Other icons can be found by referring to fontawesome.pdf (supplied with the template) and using the word after \fa in the command for the icon you want
	\icon{MapMarker}{12}{\href{https://www.openstreetmap.org/\#map=15/50.5969/4.3228}{Nivelles}}\\
	\icon{At}{12}{\href{mailto:pol.dellaiera@protonmail.com}{pol.dellaiera@protonmail.com}}\\
	\icon{Globe}{12}{\href{https://not-a-number.io}{not-a-number.io}}\\
\end{minipage}
\begin{minipage}[t]{0.20\textwidth} % 27.5% of the page width for the second row of icons
	\vspace{-\baselineskip} % Required for vertically aligning minipages

	% The first parameter is the FontAwesome icon name, the second is the box size and the third is the text
	% Other icons can be found by referring to fontawesome.pdf (supplied with the template) and using the word after \fa in the command for the icon you want
	\icon{Github}{12}{\href{https://github.com/loophp}{github.com/loophp}}\\
	\icon{Github}{12}{\href{https://github.com/drupol}{github.com/drupol}}\\
	\icon{Mastodon}{12}{\href{https://mathstodon.xyz/@Pol}{@Pol}}\\
\end{minipage}

%----------------------------------------------------------------------------------------
%	INTRODUCTION, SKILLS AND TECHNOLOGIES
%----------------------------------------------------------------------------------------
\cvsect{About}

\begin{minipage}[t]{.4\textwidth} % 40% of the page width for the introduction text
	\vspace{-\baselineskip} % Required for vertically aligning minipages
    \justify{
    Since beginning my web development journey in 2010, I have acquired a wealth
    of experience across diverse environments, including innovative start-ups
    and established consultancies.
    A highly motivated, self-taught professional, I am passionate about solving
    intricate problems by implementing elegant, streamlined solutions.
    \\\\My insatiable curiosity and meticulous nature have made me a perpetual
    learner, constantly striving to expand my knowledge.
    \\\\I take great satisfaction in creating simple, natural, and efficient
    solutions that harmoniously balance aesthetics and functionality.
    }
\end{minipage}
\hfill % Whitespace between
\begin{minipage}[t]{0.57\textwidth} % 50% of the page for the skills bar chart
	\vspace{-\baselineskip} % Required for vertically aligning minipages
	\begin{barchart}{6}
		\baritem{Linux/NixOS}{97}
        \baritem{\textbf{O}bject \textbf{O}riented \textbf{P}rogramming}{90}
		\baritem{PHP/Python}{87}
        \baritem{\textbf{F}unctional \textbf{P}rogramming}{85}
		\baritem{Git/Jujutsu}{85}
		\baritem{Algorithm}{85}
		\baritem{Docker}{75}
		\baritem{Typst}{70}
		\baritem{LaTeX}{65}
	\end{barchart}
\end{minipage}
%----------------------------------------------------------------------------------------
%	EXPERIENCE
%----------------------------------------------------------------------------------------
\cvsect{Experience}
\begin{entrylist}
	\entry
		{6/2024 -- present}
        {Full time}
		{Senior Application architect}
        {\href{https://ec.europa.eu/}{European Commission}}
        {
            Hired on behalf of a consultancy company, I am currently working at
            \href{https://ec.europa.eu/info/departments/informatics_en}{Digit} B.4 (\textit{Software Engineering Capabilities}).
			In this role, as a Senior Architect, I am developing the ECGPT application using Python and MongoDB,
			focusing on building scalable and efficient solutions.
			\\ \texttt{Python}\slashsep\texttt{MongoDB}\slashsep\texttt{Nix}\slashsep\texttt{Git}\slashsep\texttt{Jujutsu}\slashsep\texttt{InfrastructureAsCode}
        }

	\entry
		{7/2019 -- present}
        {Full time}
		{Application architect}
        {\href{https://ec.europa.eu/}{European Commission}}
        {
            Hired on behalf of a consultancy company, I am currently working at
            \href{https://ec.europa.eu/info/departments/informatics_en}{Digit} B.4 (\textit{Software Engineering Capabilities}),
            where I work in the Developer's Journey team. In this role, I guide teams and clients
            through the migration process from ColdFusion to PHP, develop and provide custom
            bundles and packages, and offer tailored assistance for diverse client needs through
            short-term consultancy missions. Additionally, I design and implement authentication libraries
            solutions and the necessary development infrastructure for multiple teams, with
            a focus on creating reproducible and ephemeral development environments based on Nix.
            My efforts have successfully facilitated seamless migrations, developed robust custom
            software solutions, and significantly enhanced team productivity through efficient,
            scalable infrastructure implementations.
			\\ \texttt{Nix}\slashsep\texttt{Symfony}\slashsep\texttt{Doctrine}\slashsep\texttt{API Platform}\slashsep\texttt{Oracle}\slashsep\texttt{Docker}\slashsep\texttt{InfrastructureAsCode}
        }

    \entry
        {\color{black!50}Before 6/2015}
        {}
        {}
        {}
        {\footnotesize{\color{black!50}This is the public and short version of my CV. Please ask the full version if needed by sending an \href{mailto:pol.dellaiera@protonmail.com}{email}.}}
\end{entrylist}

%----------------------------------------------------------------------------------------
%	EDUCATION
%----------------------------------------------------------------------------------------

\begin{minipage}[t]{.48\textwidth}
	\cvsect{Education}

    \begin{entrylist}
		\entry
			{\tiny{2021 -- 2024}}
            {Cum Laude}
			{MSc Computer Science}
            {\href{https://www.umons.be/}{Université de Mons}}
			{Thesis: "\href{https://doi.org/10.5281/zenodo.12666898}{Reproducibility in Software Engineering}}"
	\end{entrylist}

	\begin{entrylist}
		\entry
			{\tiny{2001 -- 2005}}
            {Cum Laude}
			{BSc Computer Science}
            {\href{https://www.heh.be/}{Haute Ecoles en Hainaut}}
			{IT and systems, specialisation in network and telecommunications}
	\end{entrylist}

	\begin{entrylist}
		\entry
			{\tiny{2018 -- 2021}}
            {}
			{Music theory}
            {\href{https://academiedenivelles.be/}{Académie de musique de Nivelles}}
			{Musical instrument: piano}
	\end{entrylist}
\end{minipage}
\hfill
\begin{minipage}[t]{.48\textwidth}
	\cvsect{Certificates}

	\begin{entrylist}
		\entry
			{01/2020}
            {}
			{Certificate}
			{The Linux Foundation}
			{\href{https://courses.edx.org/certificates/01fdb9d9242546e8bc45153468dfd785}{Blockchain: Understanding Its Uses and Implications}}
	\end{entrylist}

	\begin{entrylist}
		\entry
			{09/2015}
            {}
			{Certificate}
			{Acquia}
			{\href{https://certification.acquia.com/user/249}{Acquia Certified Developer}}
	\end{entrylist}

	\begin{entrylist}
		\entry
			{09/2015}
            {}
			{Certificate}
			{Acquia}
			{\href{https://certification.acquia.com/user/249}{Acquia Certified Back End Specialist}}
	\end{entrylist}
\end{minipage}


%----------------------------------------------------------------------------------------
%	ADDITIONAL INFORMATION
%----------------------------------------------------------------------------------------

\begin{minipage}[t]{.2\textwidth}
	\vspace{-\baselineskip} % Required for vertically aligning minipages

	\cvsect{Languages}

	\textbf{French} - native\\
	\textbf{Anglais} - proficient\\
	\textbf{Italian} - proficient\\
	\textbf{Dutch} - rudimentary\\
\end{minipage}
\hfill
\begin{minipage}[t]{.45\textwidth}
	\vspace{-\baselineskip} % Required for vertically aligning minipages

	\cvsect{Hobbies}

    Besides my work and the geek stuff, I’m currently fulfilling a childhood dream, I’m learning music and piano!
    I love photography and I learned by myself most of the secrets of a reflex camera, just for fun.
    I swim a lot and I also really like riding my mountain bike.
\end{minipage}
\hfill
\begin{minipage}[t]{.30\textwidth}
	\vspace{-\baselineskip} % Required for vertically aligning minipages

	\cvsect{Non profit}

    Contributor in many open-source projects, especially Nix.

    I'm also an OpenStreetMap user and contributor.
\end{minipage}
\hfill
\begin{minipage}[t]{\textwidth}
	\vspace{-\baselineskip} % Required for vertically aligning minipages

	\cvsect{Favorite quotes}

    \begin{itemize}
        \item "Simplicity is the ultimate sophistication."{\tiny{---Leonardo Da Vinci}}
        \item "Only when the last tree has been cut down, the last fish been caught, and the last stream poisoned, will we realize we cannot eat money."{\tiny{---Indian author}}
        \item "We may regard the present state of the universe as the effect of its past and the cause of its future. An intellect which at any given moment knew all of the forces that animate nature and the mutual positions of the beings that compose it, if this intellect were vast enough to submit the data to analysis, could condense into a single formula the movement of the greatest bodies of the universe and that of the lightest atom; for such an intellect nothing could be uncertain and the future just like the past would be present before its eyes."{\tiny{---Pierre Simon de Laplace}}
    \end{itemize}
\end{minipage}

%----------------------------------------------------------------------------------------

\null
\vspace*{\fill}

\noindent{\color{black!15}\rule{\textwidth}{.5pt}}

\makebox[\linewidth][r]{
    \color{black!50}
    \tiny{
        Public version
        $\cdot$
        \href{https://github.com/drupol/cv/commit/\OTPversion}{\faIcon{code-branch} {\OTPversion}}
    }
}

\end{document}
